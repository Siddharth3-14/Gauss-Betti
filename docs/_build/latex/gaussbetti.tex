%% Generated by Sphinx.
\def\sphinxdocclass{report}
\documentclass[letterpaper,10pt,english]{sphinxmanual}
\ifdefined\pdfpxdimen
   \let\sphinxpxdimen\pdfpxdimen\else\newdimen\sphinxpxdimen
\fi \sphinxpxdimen=.75bp\relax
\ifdefined\pdfimageresolution
    \pdfimageresolution= \numexpr \dimexpr1in\relax/\sphinxpxdimen\relax
\fi
%% let collapsable pdf bookmarks panel have high depth per default
\PassOptionsToPackage{bookmarksdepth=5}{hyperref}

\PassOptionsToPackage{warn}{textcomp}
\usepackage[utf8]{inputenc}
\ifdefined\DeclareUnicodeCharacter
% support both utf8 and utf8x syntaxes
  \ifdefined\DeclareUnicodeCharacterAsOptional
    \def\sphinxDUC#1{\DeclareUnicodeCharacter{"#1}}
  \else
    \let\sphinxDUC\DeclareUnicodeCharacter
  \fi
  \sphinxDUC{00A0}{\nobreakspace}
  \sphinxDUC{2500}{\sphinxunichar{2500}}
  \sphinxDUC{2502}{\sphinxunichar{2502}}
  \sphinxDUC{2514}{\sphinxunichar{2514}}
  \sphinxDUC{251C}{\sphinxunichar{251C}}
  \sphinxDUC{2572}{\textbackslash}
\fi
\usepackage{cmap}
\usepackage[T1]{fontenc}
\usepackage{amsmath,amssymb,amstext}
\usepackage{babel}



\usepackage{tgtermes}
\usepackage{tgheros}
\renewcommand{\ttdefault}{txtt}



\usepackage[Bjarne]{fncychap}
\usepackage{sphinx}

\fvset{fontsize=auto}
\usepackage{geometry}


% Include hyperref last.
\usepackage{hyperref}
% Fix anchor placement for figures with captions.
\usepackage{hypcap}% it must be loaded after hyperref.
% Set up styles of URL: it should be placed after hyperref.
\urlstyle{same}


\usepackage{sphinxmessages}
\setcounter{tocdepth}{1}



\title{GaussBetti}
\date{Jul 27, 2021}
\release{2.0.0}
\author{Siddharth}
\newcommand{\sphinxlogo}{\vbox{}}
\renewcommand{\releasename}{Release}
\makeindex
\begin{document}

\pagestyle{empty}
\sphinxmaketitle
\pagestyle{plain}
\sphinxtableofcontents
\pagestyle{normal}
\phantomsection\label{\detokenize{index::doc}}



\chapter{topologicalFunc}
\label{\detokenize{topologicalFunc:topologicalfunc}}\label{\detokenize{topologicalFunc:id1}}\label{\detokenize{topologicalFunc::doc}}\phantomsection\label{\detokenize{topologicalFunc:module-topologicalFunc}}\index{module@\spxentry{module}!topologicalFunc@\spxentry{topologicalFunc}}\index{topologicalFunc@\spxentry{topologicalFunc}!module@\spxentry{module}}\index{GaussianFiltration() (in module topologicalFunc)@\spxentry{GaussianFiltration()}\spxextra{in module topologicalFunc}}

\begin{fulllineitems}
\phantomsection\label{\detokenize{topologicalFunc:topologicalFunc.GaussianFiltration}}\pysiglinewithargsret{\sphinxcode{\sphinxupquote{topologicalFunc.}}\sphinxbfcode{\sphinxupquote{GaussianFiltration}}}{\emph{\DUrole{n}{GaussianRandomField}}, \emph{\DUrole{n}{type}\DUrole{o}{=}\DUrole{default_value}{\textquotesingle{}lower\textquotesingle{}}}}{}
\sphinxAtStartPar
Generates Filtration for the Gaussian Random Field.
\begin{quote}\begin{description}
\item[{Parameters}] \leavevmode\begin{itemize}
\item {} 
\sphinxAtStartPar
\sphinxstyleliteralstrong{\sphinxupquote{GaussianRandomField}} (\sphinxstyleliteralemphasis{\sphinxupquote{array}}) \textendash{} numpy 2\sphinxhyphen{}D array. The Gaussian Random Field generated from the class using Gen\_GRF method.

\item {} 
\sphinxAtStartPar
\sphinxstyleliteralstrong{\sphinxupquote{type}} (\sphinxstyleliteralemphasis{\sphinxupquote{string}}) \textendash{} Takes ipnut either ‘lower’ or ‘upper’ for lower or upper filtration.

\item {} 
\sphinxAtStartPar
\sphinxstyleliteralstrong{\sphinxupquote{nsize}} (\sphinxstyleliteralemphasis{\sphinxupquote{integer}}) \textendash{} Size of the Gaussian Random Fields grid.

\end{itemize}

\item[{Returns}] \leavevmode
\sphinxAtStartPar
Filtration Diagram

\item[{Return type}] \leavevmode
\sphinxAtStartPar
Dionysus object

\end{description}\end{quote}

\end{fulllineitems}

\index{GenerateBettiP() (in module topologicalFunc)@\spxentry{GenerateBettiP()}\spxextra{in module topologicalFunc}}

\begin{fulllineitems}
\phantomsection\label{\detokenize{topologicalFunc:topologicalFunc.GenerateBettiP}}\pysiglinewithargsret{\sphinxcode{\sphinxupquote{topologicalFunc.}}\sphinxbfcode{\sphinxupquote{GenerateBettiP}}}{\emph{\DUrole{n}{Filtraion}}, \emph{\DUrole{n}{thresholds\_start}}, \emph{\DUrole{n}{thresholds\_stop}}, \emph{\DUrole{n}{type}\DUrole{o}{=}\DUrole{default_value}{\textquotesingle{}lower\textquotesingle{}}}}{}
\sphinxAtStartPar
Generates the Betti numbers from the Filtration diagram.
\begin{quote}\begin{description}
\item[{Parameters}] \leavevmode\begin{itemize}
\item {} 
\sphinxAtStartPar
\sphinxstyleliteralstrong{\sphinxupquote{Filtration}} (\sphinxstyleliteralemphasis{\sphinxupquote{Dionysus object}}) \textendash{} Output of GaussianFiltration.

\item {} 
\sphinxAtStartPar
\sphinxstyleliteralstrong{\sphinxupquote{thresholds\_start}} (\sphinxstyleliteralemphasis{\sphinxupquote{float}}) \textendash{} start value for generating superlevels of the Gaussian Random field .

\item {} 
\sphinxAtStartPar
\sphinxstyleliteralstrong{\sphinxupquote{thresholds\_stop}} (\sphinxstyleliteralemphasis{\sphinxupquote{float}}) \textendash{} stop value for generating superlevels of the Gaussian Random field.

\end{itemize}

\item[{Returns}] \leavevmode
\sphinxAtStartPar
Multidimensaion array contaiing Betti numbers for different dimensions

\item[{Return type}] \leavevmode
\sphinxAtStartPar
Numpy array

\end{description}\end{quote}

\end{fulllineitems}

\index{GenerateGenus() (in module topologicalFunc)@\spxentry{GenerateGenus()}\spxextra{in module topologicalFunc}}

\begin{fulllineitems}
\phantomsection\label{\detokenize{topologicalFunc:topologicalFunc.GenerateGenus}}\pysiglinewithargsret{\sphinxcode{\sphinxupquote{topologicalFunc.}}\sphinxbfcode{\sphinxupquote{GenerateGenus}}}{\emph{\DUrole{n}{Betti\_array}}}{}
\sphinxAtStartPar
Generates the Genus curve for gaussian random field using Betti arrays.
\begin{quote}\begin{description}
\item[{Parameters}] \leavevmode
\sphinxAtStartPar
\sphinxstyleliteralstrong{\sphinxupquote{array}} (\sphinxstyleliteralemphasis{\sphinxupquote{Betti}}) \textendash{} Betti array from GenerateBettiP.

\item[{Returns}] \leavevmode
\sphinxAtStartPar
1\sphinxhyphen{}D array contaiing Genus curve for the Gaussian random field.

\item[{Return type}] \leavevmode
\sphinxAtStartPar
Numpy array

\end{description}\end{quote}

\end{fulllineitems}



\chapter{Utilities}
\label{\detokenize{utilities:utilities}}\label{\detokenize{utilities:id1}}\label{\detokenize{utilities::doc}}\phantomsection\label{\detokenize{utilities:module-utilities}}\index{module@\spxentry{module}!utilities@\spxentry{utilities}}\index{utilities@\spxentry{utilities}!module@\spxentry{module}}\index{Generate\_BettiGenus\_array() (in module utilities)@\spxentry{Generate\_BettiGenus\_array()}\spxextra{in module utilities}}

\begin{fulllineitems}
\phantomsection\label{\detokenize{utilities:utilities.Generate_BettiGenus_array}}\pysiglinewithargsret{\sphinxcode{\sphinxupquote{utilities.}}\sphinxbfcode{\sphinxupquote{Generate\_BettiGenus\_array}}}{\emph{\DUrole{n}{Nsize}}, \emph{\DUrole{n}{power\_index\_null}}, \emph{\DUrole{n}{power\_index\_test}}, \emph{\DUrole{n}{average}}, \emph{\DUrole{n}{iteration}}, \emph{\DUrole{n}{filtration\_threshold\_start}}, \emph{\DUrole{n}{filtration\_threshold\_stop}}, \emph{\DUrole{n}{type1}\DUrole{o}{=}\DUrole{default_value}{\textquotesingle{}lower\textquotesingle{}}}}{}
\sphinxAtStartPar
Generate\_Likelihood\_Array

\sphinxAtStartPar
Generates the Betti and Genus curves for specified parameters.
\begin{quote}\begin{description}
\item[{Parameters}] \leavevmode\begin{itemize}
\item {} 
\sphinxAtStartPar
\sphinxstyleliteralstrong{\sphinxupquote{Nsize}} (\sphinxstyleliteralemphasis{\sphinxupquote{integer}}) \textendash{} grid size of the Gaussian Random Field

\item {} 
\sphinxAtStartPar
\sphinxstyleliteralstrong{\sphinxupquote{power\_index\_null}} (\sphinxstyleliteralemphasis{\sphinxupquote{float}}) \textendash{} Power spectral index of Null Hypothesis

\item {} 
\sphinxAtStartPar
\sphinxstyleliteralstrong{\sphinxupquote{power\_index\_test}} (\sphinxstyleliteralemphasis{\sphinxupquote{float}}) \textendash{} Power spectral index of Test Hypothesis

\item {} 
\sphinxAtStartPar
\sphinxstyleliteralstrong{\sphinxupquote{average}} (\sphinxstyleliteralemphasis{\sphinxupquote{integer}}) \textendash{} No. of times the betti curves need to be averaged

\item {} 
\sphinxAtStartPar
\sphinxstyleliteralstrong{\sphinxupquote{iteration}} (\sphinxstyleliteralemphasis{\sphinxupquote{integer}}) \textendash{} Size of the arrays generated

\item {} 
\sphinxAtStartPar
\sphinxstyleliteralstrong{\sphinxupquote{filtration\_threshold\_start}} (\sphinxstyleliteralemphasis{\sphinxupquote{float}}) \textendash{} Start value for generating filtraion from dionysus

\item {} 
\sphinxAtStartPar
\sphinxstyleliteralstrong{\sphinxupquote{filtration\_threshold\_stop}} (\sphinxstyleliteralemphasis{\sphinxupquote{float}}) \textendash{} Stop value for generation filtration from dionysus

\item {} 
\sphinxAtStartPar
\sphinxstyleliteralstrong{\sphinxupquote{type}} \textendash{} Type of filtration accepted values are ‘lower’ ‘upper

\end{itemize}

\item[{Returns}] \leavevmode
\sphinxAtStartPar
array of Betti and Genus curves

\item[{Return type}] \leavevmode
\sphinxAtStartPar
numpy array

\end{description}\end{quote}

\end{fulllineitems}

\index{Generate\_Likelihood\_Array() (in module utilities)@\spxentry{Generate\_Likelihood\_Array()}\spxextra{in module utilities}}

\begin{fulllineitems}
\phantomsection\label{\detokenize{utilities:utilities.Generate_Likelihood_Array}}\pysiglinewithargsret{\sphinxcode{\sphinxupquote{utilities.}}\sphinxbfcode{\sphinxupquote{Generate\_Likelihood\_Array}}}{\emph{\DUrole{n}{Nsize}}, \emph{\DUrole{n}{power\_index\_null}}, \emph{\DUrole{n}{power\_index\_test}}, \emph{\DUrole{n}{iteration}}}{}
\sphinxAtStartPar
Generates the array of likelihood ratios for making ROC curves.
\begin{quote}\begin{description}
\item[{Parameters}] \leavevmode\begin{itemize}
\item {} 
\sphinxAtStartPar
\sphinxstyleliteralstrong{\sphinxupquote{Nsize}} (\sphinxstyleliteralemphasis{\sphinxupquote{integer}}) \textendash{} grid size of the Gaussian Random Field

\item {} 
\sphinxAtStartPar
\sphinxstyleliteralstrong{\sphinxupquote{power\_index\_null}} (\sphinxstyleliteralemphasis{\sphinxupquote{float}}) \textendash{} Power spectral index of Null Hypothesis

\item {} 
\sphinxAtStartPar
\sphinxstyleliteralstrong{\sphinxupquote{power\_index\_test}} (\sphinxstyleliteralemphasis{\sphinxupquote{float}}) \textendash{} Power spectral index of Test Hypothesis

\item {} 
\sphinxAtStartPar
\sphinxstyleliteralstrong{\sphinxupquote{iteration}} (\sphinxstyleliteralemphasis{\sphinxupquote{integer}}) \textendash{} Size of the likelihood ratio array generated

\end{itemize}

\item[{Returns}] \leavevmode
\sphinxAtStartPar
array of likelihood ratios

\item[{Return type}] \leavevmode
\sphinxAtStartPar
numpy array

\end{description}\end{quote}

\end{fulllineitems}

\index{KLdivergence() (in module utilities)@\spxentry{KLdivergence()}\spxextra{in module utilities}}

\begin{fulllineitems}
\phantomsection\label{\detokenize{utilities:utilities.KLdivergence}}\pysiglinewithargsret{\sphinxcode{\sphinxupquote{utilities.}}\sphinxbfcode{\sphinxupquote{KLdivergence}}}{\emph{\DUrole{n}{x}}, \emph{\DUrole{n}{y1}}, \emph{\DUrole{n}{y2}}}{}
\sphinxAtStartPar
Calculates the KL divergence for 2 different Gaussian Random Field.
\begin{quote}\begin{description}
\item[{Parameters}] \leavevmode\begin{itemize}
\item {} 
\sphinxAtStartPar
\sphinxstyleliteralstrong{\sphinxupquote{x}} (\sphinxstyleliteralemphasis{\sphinxupquote{array}}) \textendash{} 

\item {} 
\sphinxAtStartPar
\sphinxstyleliteralstrong{\sphinxupquote{y1}} (\sphinxstyleliteralemphasis{\sphinxupquote{array}}) \textendash{} Gausian Random Field of null hypothesis as a 1\sphinxhyphen{}D array

\item {} 
\sphinxAtStartPar
\sphinxstyleliteralstrong{\sphinxupquote{y2}} (\sphinxstyleliteralemphasis{\sphinxupquote{array}}) \textendash{} Gaussian Random Field of test hypothesis as a 1\sphinxhyphen{}D array

\end{itemize}

\item[{Returns}] \leavevmode
\sphinxAtStartPar
KL divergence

\item[{Return type}] \leavevmode
\sphinxAtStartPar
float

\end{description}\end{quote}

\end{fulllineitems}

\index{plotROC() (in module utilities)@\spxentry{plotROC()}\spxextra{in module utilities}}

\begin{fulllineitems}
\phantomsection\label{\detokenize{utilities:utilities.plotROC}}\pysiglinewithargsret{\sphinxcode{\sphinxupquote{utilities.}}\sphinxbfcode{\sphinxupquote{plotROC}}}{\emph{\DUrole{n}{PFA}}, \emph{\DUrole{n}{PD}}, \emph{\DUrole{n}{nsize}}, \emph{\DUrole{n}{num\_iter}}, \emph{\DUrole{n}{H0}}, \emph{\DUrole{n}{H1}}, \emph{\DUrole{n}{type1}}, \emph{\DUrole{n}{Betti}\DUrole{o}{=}\DUrole{default_value}{\textquotesingle{}default\textquotesingle{}}}}{}
\sphinxAtStartPar
Plots the PFA and PD ROC graph with the labels provided through parameters.
\begin{quote}\begin{description}
\item[{Parameters}] \leavevmode\begin{itemize}
\item {} 
\sphinxAtStartPar
\sphinxstyleliteralstrong{\sphinxupquote{PFA}} (\sphinxstyleliteralemphasis{\sphinxupquote{array}}) \textendash{} numpy vector. The PFA array generated during ROC gen.

\item {} 
\sphinxAtStartPar
\sphinxstyleliteralstrong{\sphinxupquote{PD}} (\sphinxstyleliteralemphasis{\sphinxupquote{array}}) \textendash{} numpy vector. THe PD array generated during ROC gen.

\item {} 
\sphinxAtStartPar
\sphinxstyleliteralstrong{\sphinxupquote{nsize}} (\sphinxstyleliteralemphasis{\sphinxupquote{integer}}) \textendash{} Size of the Gaussian Random Fields grid.

\item {} 
\sphinxAtStartPar
\sphinxstyleliteralstrong{\sphinxupquote{num\_iter}} (\sphinxstyleliteralemphasis{\sphinxupquote{integer}}) \textendash{} Number of iteration for which ROC gen is run.

\item {} 
\sphinxAtStartPar
\sphinxstyleliteralstrong{\sphinxupquote{H0}} (\sphinxstyleliteralemphasis{\sphinxupquote{float}}) \textendash{} Power spectral index of Null Hypothesis.

\item {} 
\sphinxAtStartPar
\sphinxstyleliteralstrong{\sphinxupquote{H1}} (\sphinxstyleliteralemphasis{\sphinxupquote{float}}) \textendash{} Power spectral index of Test Hypothesis.

\item {} 
\sphinxAtStartPar
\sphinxstyleliteralstrong{\sphinxupquote{type1}} (\sphinxstyleliteralemphasis{\sphinxupquote{string}}) \textendash{} type1 of the ROC curve generated takes value ‘likelihood’,’betti’,’genus

\item {} 
\sphinxAtStartPar
\sphinxstyleliteralstrong{\sphinxupquote{Betti}} (\sphinxstyleliteralemphasis{\sphinxupquote{integer}}) \textendash{} Dimension of Betti curve not needed when type = likelihood

\end{itemize}

\item[{Returns}] \leavevmode
\sphinxAtStartPar
None

\item[{Return type}] \leavevmode
\sphinxAtStartPar
None

\end{description}\end{quote}

\end{fulllineitems}

\index{readROC() (in module utilities)@\spxentry{readROC()}\spxextra{in module utilities}}

\begin{fulllineitems}
\phantomsection\label{\detokenize{utilities:utilities.readROC}}\pysiglinewithargsret{\sphinxcode{\sphinxupquote{utilities.}}\sphinxbfcode{\sphinxupquote{readROC}}}{\emph{\DUrole{n}{nsize}}, \emph{\DUrole{n}{num\_iter}}, \emph{\DUrole{n}{H0}}, \emph{\DUrole{n}{H1}}, \emph{\DUrole{n}{type1}}, \emph{\DUrole{n}{Betti}\DUrole{o}{=}\DUrole{default_value}{\textquotesingle{}default\textquotesingle{}}}}{}
\sphinxAtStartPar
Reads the PFA and PD array from the files generated using saveROC.
\begin{quote}\begin{description}
\item[{Parameters}] \leavevmode\begin{itemize}
\item {} 
\sphinxAtStartPar
\sphinxstyleliteralstrong{\sphinxupquote{nsize}} (\sphinxstyleliteralemphasis{\sphinxupquote{integer}}) \textendash{} Size of the Gaussian Random Fields grid.

\item {} 
\sphinxAtStartPar
\sphinxstyleliteralstrong{\sphinxupquote{num\_iter}} (\sphinxstyleliteralemphasis{\sphinxupquote{integer}}) \textendash{} Number of iteration for which ROC gen is run.

\item {} 
\sphinxAtStartPar
\sphinxstyleliteralstrong{\sphinxupquote{H0}} (\sphinxstyleliteralemphasis{\sphinxupquote{integer}}) \textendash{} Power spectral index of Null Hypothesis.

\item {} 
\sphinxAtStartPar
\sphinxstyleliteralstrong{\sphinxupquote{H1}} (\sphinxstyleliteralemphasis{\sphinxupquote{integer}}) \textendash{} Power spectral index of Test Hypothesis.

\item {} 
\sphinxAtStartPar
\sphinxstyleliteralstrong{\sphinxupquote{type1}} (\sphinxstyleliteralemphasis{\sphinxupquote{string}}) \textendash{} type1 of the ROC curve generated takes value ‘likelihood’,’betti’,’genus

\item {} 
\sphinxAtStartPar
\sphinxstyleliteralstrong{\sphinxupquote{Betti}} (\sphinxstyleliteralemphasis{\sphinxupquote{integer}}) \textendash{} Dimension of Betti curve not needed when type = likelihood

\end{itemize}

\item[{Returns}] \leavevmode
\sphinxAtStartPar
Returns PFA and PD arrays

\item[{Return type}] \leavevmode
\sphinxAtStartPar
Numpy Array

\end{description}\end{quote}

\end{fulllineitems}

\index{saveROC() (in module utilities)@\spxentry{saveROC()}\spxextra{in module utilities}}

\begin{fulllineitems}
\phantomsection\label{\detokenize{utilities:utilities.saveROC}}\pysiglinewithargsret{\sphinxcode{\sphinxupquote{utilities.}}\sphinxbfcode{\sphinxupquote{saveROC}}}{\emph{\DUrole{n}{PFA}}, \emph{\DUrole{n}{PD}}, \emph{\DUrole{n}{nsize}}, \emph{\DUrole{n}{num\_iter}}, \emph{\DUrole{n}{H0}}, \emph{\DUrole{n}{H1}}, \emph{\DUrole{n}{type1}}, \emph{\DUrole{n}{Betti}\DUrole{o}{=}\DUrole{default_value}{\textquotesingle{}default\textquotesingle{}}}}{}
\sphinxAtStartPar
SaveROC

\sphinxAtStartPar
Saves the PFA and PD array with the labels provided through parameters.
\begin{quote}\begin{description}
\item[{Parameters}] \leavevmode\begin{itemize}
\item {} 
\sphinxAtStartPar
\sphinxstyleliteralstrong{\sphinxupquote{PFA}} (\sphinxstyleliteralemphasis{\sphinxupquote{array}}) \textendash{} numpy vector. The PFA array generated during ROC gen.

\item {} 
\sphinxAtStartPar
\sphinxstyleliteralstrong{\sphinxupquote{PD}} (\sphinxstyleliteralemphasis{\sphinxupquote{array}}) \textendash{} numpy vector. THe PD array generated during ROC gen.

\item {} 
\sphinxAtStartPar
\sphinxstyleliteralstrong{\sphinxupquote{nsize}} (\sphinxstyleliteralemphasis{\sphinxupquote{integer}}) \textendash{} Size of the Gaussian Random Fields grid.

\item {} 
\sphinxAtStartPar
\sphinxstyleliteralstrong{\sphinxupquote{num\_iter}} (\sphinxstyleliteralemphasis{\sphinxupquote{integer}}) \textendash{} Number of iteration for which ROC gen is run.

\item {} 
\sphinxAtStartPar
\sphinxstyleliteralstrong{\sphinxupquote{H0}} (\sphinxstyleliteralemphasis{\sphinxupquote{float}}) \textendash{} Power spectral index of Null Hypothesis.

\item {} 
\sphinxAtStartPar
\sphinxstyleliteralstrong{\sphinxupquote{H1}} (\sphinxstyleliteralemphasis{\sphinxupquote{float}}) \textendash{} Power spectral index of Test Hypothesis.

\item {} 
\sphinxAtStartPar
\sphinxstyleliteralstrong{\sphinxupquote{type1}} (\sphinxstyleliteralemphasis{\sphinxupquote{string}}) \textendash{} type1 of the ROC curve generated takes value ‘likelihood’,’betti’,’genus

\item {} 
\sphinxAtStartPar
\sphinxstyleliteralstrong{\sphinxupquote{Betti}} (\sphinxstyleliteralemphasis{\sphinxupquote{integer}}) \textendash{} Dimension of Betti curve not needed when type = likelihood

\end{itemize}

\item[{Returns}] \leavevmode
\sphinxAtStartPar
None

\item[{Return type}] \leavevmode
\sphinxAtStartPar
None

\end{description}\end{quote}

\end{fulllineitems}



\chapter{gaussClass}
\label{\detokenize{gaussClass:gaussclass}}\label{\detokenize{gaussClass:id1}}\label{\detokenize{gaussClass::doc}}\phantomsection\label{\detokenize{gaussClass:module-gaussClass}}\index{module@\spxentry{module}!gaussClass@\spxentry{gaussClass}}\index{gaussClass@\spxentry{gaussClass}!module@\spxentry{module}}\index{GaussianRandomField (class in gaussClass)@\spxentry{GaussianRandomField}\spxextra{class in gaussClass}}

\begin{fulllineitems}
\phantomsection\label{\detokenize{gaussClass:gaussClass.GaussianRandomField}}\pysiglinewithargsret{\sphinxbfcode{\sphinxupquote{class }}\sphinxcode{\sphinxupquote{gaussClass.}}\sphinxbfcode{\sphinxupquote{GaussianRandomField}}}{\emph{\DUrole{n}{Nsize}}, \emph{\DUrole{n}{n}}}{}
\sphinxAtStartPar
The class for making Gaussian random field with specified spectral index and size of grid.
\index{Nzise (gaussClass.GaussianRandomField attribute)@\spxentry{Nzise}\spxextra{gaussClass.GaussianRandomField attribute}}

\begin{fulllineitems}
\phantomsection\label{\detokenize{gaussClass:gaussClass.GaussianRandomField.Nzise}}\pysigline{\sphinxbfcode{\sphinxupquote{Nzise}}}
\sphinxAtStartPar
size of the grid.
\begin{quote}\begin{description}
\item[{Type}] \leavevmode
\sphinxAtStartPar
int

\end{description}\end{quote}

\end{fulllineitems}

\index{n (gaussClass.GaussianRandomField attribute)@\spxentry{n}\spxextra{gaussClass.GaussianRandomField attribute}}

\begin{fulllineitems}
\phantomsection\label{\detokenize{gaussClass:gaussClass.GaussianRandomField.n}}\pysigline{\sphinxbfcode{\sphinxupquote{n}}}
\sphinxAtStartPar
Spectral index of the power law used to generate the Gaussian Random Field.
\begin{quote}\begin{description}
\item[{Type}] \leavevmode
\sphinxAtStartPar
int

\end{description}\end{quote}

\end{fulllineitems}

\index{k\_ind (gaussClass.GaussianRandomField attribute)@\spxentry{k\_ind}\spxextra{gaussClass.GaussianRandomField attribute}}

\begin{fulllineitems}
\phantomsection\label{\detokenize{gaussClass:gaussClass.GaussianRandomField.k_ind}}\pysigline{\sphinxbfcode{\sphinxupquote{k\_ind}}}
\sphinxAtStartPar
Grid in the fourier space.
\begin{quote}\begin{description}
\item[{Type}] \leavevmode
\sphinxAtStartPar
array

\end{description}\end{quote}

\end{fulllineitems}

\index{PowerSpectrum (gaussClass.GaussianRandomField attribute)@\spxentry{PowerSpectrum}\spxextra{gaussClass.GaussianRandomField attribute}}

\begin{fulllineitems}
\phantomsection\label{\detokenize{gaussClass:gaussClass.GaussianRandomField.PowerSpectrum}}\pysigline{\sphinxbfcode{\sphinxupquote{PowerSpectrum}}}
\sphinxAtStartPar
The power spectrum grid made using the spectral index used to make the Gaussian Random Field.
\begin{quote}\begin{description}
\item[{Type}] \leavevmode
\sphinxAtStartPar
array

\end{description}\end{quote}

\end{fulllineitems}

\index{corr\_s (gaussClass.GaussianRandomField attribute)@\spxentry{corr\_s}\spxextra{gaussClass.GaussianRandomField attribute}}

\begin{fulllineitems}
\phantomsection\label{\detokenize{gaussClass:gaussClass.GaussianRandomField.corr_s}}\pysigline{\sphinxbfcode{\sphinxupquote{corr\_s}}}
\sphinxAtStartPar
Correlation matrix in the fourier space.
\begin{quote}\begin{description}
\item[{Type}] \leavevmode
\sphinxAtStartPar
array

\end{description}\end{quote}

\end{fulllineitems}

\index{corr\_f (gaussClass.GaussianRandomField attribute)@\spxentry{corr\_f}\spxextra{gaussClass.GaussianRandomField attribute}}

\begin{fulllineitems}
\phantomsection\label{\detokenize{gaussClass:gaussClass.GaussianRandomField.corr_f}}\pysigline{\sphinxbfcode{\sphinxupquote{corr\_f}}}
\sphinxAtStartPar
Correlation matrix in the spatial space.
\begin{quote}\begin{description}
\item[{Type}] \leavevmode
\sphinxAtStartPar
array

\end{description}\end{quote}

\end{fulllineitems}

\index{Gen\_GRF() (gaussClass.GaussianRandomField method)@\spxentry{Gen\_GRF()}\spxextra{gaussClass.GaussianRandomField method}}

\begin{fulllineitems}
\phantomsection\label{\detokenize{gaussClass:gaussClass.GaussianRandomField.Gen_GRF}}\pysiglinewithargsret{\sphinxbfcode{\sphinxupquote{Gen\_GRF}}}{\emph{\DUrole{n}{type}\DUrole{o}{=}\DUrole{default_value}{\textquotesingle{}grid\textquotesingle{}}}}{}
\sphinxAtStartPar
GenerateBettiP

\sphinxAtStartPar
Generates the Gaussian Random field with the specified paramters.
\begin{quote}\begin{description}
\item[{Parameters}] \leavevmode
\sphinxAtStartPar
\sphinxstyleliteralstrong{\sphinxupquote{type}} (\sphinxstyleliteralemphasis{\sphinxupquote{str}}) \textendash{} Takes either ‘grid’ or ‘array’ in string format

\end{description}\end{quote}

\sphinxAtStartPar
Returns:
Numpy array: Gaussian Random field

\end{fulllineitems}

\index{PowerSpectrum\_grid\_generator() (gaussClass.GaussianRandomField method)@\spxentry{PowerSpectrum\_grid\_generator()}\spxextra{gaussClass.GaussianRandomField method}}

\begin{fulllineitems}
\phantomsection\label{\detokenize{gaussClass:gaussClass.GaussianRandomField.PowerSpectrum_grid_generator}}\pysiglinewithargsret{\sphinxbfcode{\sphinxupquote{PowerSpectrum\_grid\_generator}}}{}{}
\sphinxAtStartPar
Generates the powerspectrum grid.

\end{fulllineitems}

\index{fourier\_space\_ind() (gaussClass.GaussianRandomField method)@\spxentry{fourier\_space\_ind()}\spxextra{gaussClass.GaussianRandomField method}}

\begin{fulllineitems}
\phantomsection\label{\detokenize{gaussClass:gaussClass.GaussianRandomField.fourier_space_ind}}\pysiglinewithargsret{\sphinxbfcode{\sphinxupquote{fourier\_space\_ind}}}{}{}
\sphinxAtStartPar
Generates the fourier space grid.

\end{fulllineitems}

\index{gen\_correlation() (gaussClass.GaussianRandomField method)@\spxentry{gen\_correlation()}\spxextra{gaussClass.GaussianRandomField method}}

\begin{fulllineitems}
\phantomsection\label{\detokenize{gaussClass:gaussClass.GaussianRandomField.gen_correlation}}\pysiglinewithargsret{\sphinxbfcode{\sphinxupquote{gen\_correlation}}}{}{}
\sphinxAtStartPar
Generates the correlation matrices in fourier and spatial spcae.

\end{fulllineitems}


\end{fulllineitems}



\chapter{Indices and tables}
\label{\detokenize{index:indices-and-tables}}\begin{itemize}
\item {} 
\sphinxAtStartPar
\DUrole{xref,std,std-ref}{genindex}

\item {} 
\sphinxAtStartPar
\DUrole{xref,std,std-ref}{modindex}

\item {} 
\sphinxAtStartPar
\DUrole{xref,std,std-ref}{search}

\end{itemize}


\renewcommand{\indexname}{Python Module Index}
\begin{sphinxtheindex}
\let\bigletter\sphinxstyleindexlettergroup
\bigletter{g}
\item\relax\sphinxstyleindexentry{gaussClass}\sphinxstyleindexpageref{gaussClass:\detokenize{module-gaussClass}}
\indexspace
\bigletter{t}
\item\relax\sphinxstyleindexentry{topologicalFunc}\sphinxstyleindexpageref{topologicalFunc:\detokenize{module-topologicalFunc}}
\indexspace
\bigletter{u}
\item\relax\sphinxstyleindexentry{utilities}\sphinxstyleindexpageref{utilities:\detokenize{module-utilities}}
\end{sphinxtheindex}

\renewcommand{\indexname}{Index}
\printindex
\end{document}